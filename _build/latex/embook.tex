%% Generated by Sphinx.
\def\sphinxdocclass{jupyterBook}
\documentclass[letterpaper,10pt,english]{jupyterBook}
\ifdefined\pdfpxdimen
   \let\sphinxpxdimen\pdfpxdimen\else\newdimen\sphinxpxdimen
\fi \sphinxpxdimen=.75bp\relax
\ifdefined\pdfimageresolution
    \pdfimageresolution= \numexpr \dimexpr1in\relax/\sphinxpxdimen\relax
\fi
%% let collapsable pdf bookmarks panel have high depth per default
\PassOptionsToPackage{bookmarksdepth=5}{hyperref}
%% turn off hyperref patch of \index as sphinx.xdy xindy module takes care of
%% suitable \hyperpage mark-up, working around hyperref-xindy incompatibility
\PassOptionsToPackage{hyperindex=false}{hyperref}
%% memoir class requires extra handling
\makeatletter\@ifclassloaded{memoir}
{\ifdefined\memhyperindexfalse\memhyperindexfalse\fi}{}\makeatother

\PassOptionsToPackage{warn}{textcomp}

\catcode`^^^^00a0\active\protected\def^^^^00a0{\leavevmode\nobreak\ }
\usepackage{cmap}
\usepackage{fontspec}
\defaultfontfeatures[\rmfamily,\sffamily,\ttfamily]{}
\usepackage{amsmath,amssymb,amstext}
\usepackage{polyglossia}
\setmainlanguage{english}



\setmainfont{FreeSerif}[
  Extension      = .otf,
  UprightFont    = *,
  ItalicFont     = *Italic,
  BoldFont       = *Bold,
  BoldItalicFont = *BoldItalic
]
\setsansfont{FreeSans}[
  Extension      = .otf,
  UprightFont    = *,
  ItalicFont     = *Oblique,
  BoldFont       = *Bold,
  BoldItalicFont = *BoldOblique,
]
\setmonofont{FreeMono}[
  Extension      = .otf,
  UprightFont    = *,
  ItalicFont     = *Oblique,
  BoldFont       = *Bold,
  BoldItalicFont = *BoldOblique,
]



\usepackage[Bjarne]{fncychap}
\usepackage[,numfigreset=1,mathnumfig]{sphinx}

\fvset{fontsize=\small}
\usepackage{geometry}


% Include hyperref last.
\usepackage{hyperref}
% Fix anchor placement for figures with captions.
\usepackage{hypcap}% it must be loaded after hyperref.
% Set up styles of URL: it should be placed after hyperref.
\urlstyle{same}


\usepackage{sphinxmessages}



        % Start of preamble defined in sphinx-jupyterbook-latex %
         \usepackage[Latin,Greek]{ucharclasses}
        \usepackage{unicode-math}
        % fixing title of the toc
        \addto\captionsenglish{\renewcommand{\contentsname}{Contents}}
        \hypersetup{
            pdfencoding=auto,
            psdextra
        }
        % End of preamble defined in sphinx-jupyterbook-latex %
        

\title{Electromagnetism}
\date{Jul 22, 2022}
\release{}
\author{JSW}
\newcommand{\sphinxlogo}{\vbox{}}
\renewcommand{\releasename}{}
\makeindex
\begin{document}

\pagestyle{empty}
\sphinxmaketitle
\pagestyle{plain}
\sphinxtableofcontents
\pagestyle{normal}
\phantomsection\label{\detokenize{intro::doc}}


\sphinxAtStartPar
教學順序
\begin{itemize}
\item {} 
\sphinxAtStartPar
Vector Analysis
\begin{itemize}
\item {} 
\sphinxAtStartPar
向量操作

\item {} 
\sphinxAtStartPar
向量微分

\item {} 
\sphinxAtStartPar
向量積分

\end{itemize}

\item {} 
\sphinxAtStartPar
Electrostatics

\item {} 
\sphinxAtStartPar
Electrostatics in Matter

\item {} 
\sphinxAtStartPar
Methods for Boundary Value Problems

\item {} 
\sphinxAtStartPar
Magnetostatics

\item {} 
\sphinxAtStartPar
Electrodynamics

\end{itemize}

\begin{DUlineblock}{0em}
\item[] \sphinxstylestrong{\Large Vectors do matter 方向很重要!}
\end{DUlineblock}
\begin{quote}

\sphinxAtStartPar
人生是一場馬拉松,你今天的努力,不會立刻改變明天的你,但你立志的方向,將決定五年後、十年後、二十年後的你。
\end{quote}
\begin{quote}

\sphinxAtStartPar
Life is accumulation. Your height \(h\) is determined by not only your talent (velocity \(\mathbf{v}\)) but also the path you walk along.
\end{quote}
\begin{equation*}
\begin{split}h = \int_{\mathrm{now}}^{\mathrm{future}} \mathbf{v}\cdot (\nabla h) dt = \int_{Path}\nabla h \cdot d\mathbf{l} \end{split}
\end{equation*}
\sphinxAtStartPar
梯度版本的微積分基本定理的啟發
\begin{quote}

\sphinxAtStartPar
高度由走過的路 \(d\mathbf{l}\) 與遇到的梯度 \(\nabla h\) 決定
\end{quote}
\begin{quote}

\sphinxAtStartPar
通往山頂的路不是唯一的
\end{quote}


\chapter{Vector Analysis}
\label{\detokenize{va/intro:vector-analysis}}\label{\detokenize{va/intro::doc}}\begin{itemize}
\item {} 
\sphinxAtStartPar
向量操作
\begin{itemize}
\item {} 
\sphinxAtStartPar
inner product

\item {} 
\sphinxAtStartPar
cross product

\end{itemize}

\item {} 
\sphinxAtStartPar
向量微分
\begin{itemize}
\item {} 
\sphinxAtStartPar
gradient

\item {} 
\sphinxAtStartPar
divergence

\item {} 
\sphinxAtStartPar
curl

\end{itemize}

\item {} 
\sphinxAtStartPar
向量積分
\begin{itemize}
\item {} 
\sphinxAtStartPar
line integral

\item {} 
\sphinxAtStartPar
surface integral

\item {} 
\sphinxAtStartPar
volume integral

\item {} 
\sphinxAtStartPar
fumdamental theorom of calculus

\item {} 
\sphinxAtStartPar
Helmholtz theorem

\end{itemize}

\end{itemize}


\section{Vector Algebras}
\label{\detokenize{va/vec_alge:vector-algebras}}\label{\detokenize{va/vec_alge::doc}}

\subsection{Notations for Vectors}
\label{\detokenize{va/vec_alge:notations-for-vectors}}
\sphinxAtStartPar
A vector is an object with an \sphinxstylestrong{amplitude} and a \sphinxstylestrong{direction}.
向量常見的符號有上面有箭頭的變數 \(\vec{A}\) 或者是粗體的變數 \(\mathbf{A}\)。
\begin{itemize}
\item {} 
\sphinxAtStartPar
分量表示法:
\(\mathbf{A} = A_x \hat{i} + A_y \hat{j} + A_z \hat{k}\)

\item {} 
\sphinxAtStartPar
\(\hat{i}\) 代表 \(x\) 方向的單位向量
\(\hat{j}\) 代表 \(y\) 方向的單位向量
\(\hat{k}\) 代表 \(z\) 方向的單位向量
有些書使用 \(\hat{x}\), \(\hat{y}\), \(\hat{z}\) 或
\(\hat{e}_x\), \(\hat{e}_y\), \(\hat{e}_z\) 代表單位向量

\item {} 
\sphinxAtStartPar
The amplitude of \(\mathbf{A}\) is

\end{itemize}
\begin{equation*}
\begin{split}|\mathbf{A}| = A = \sqrt{A_x^2+A_y^2+A_z^2} \end{split}
\end{equation*}

\subsection{Tensors}
\label{\detokenize{va/vec_alge:tensors}}
\sphinxAtStartPar
我們也可以用張量的寫法來表示向量。張量可視為更廣義的向量,用來表示更高維的 arrays。沒有 index 的數,我們稱為純量 scalar。有一個index的數稱為向量,例如 \(A_\mu\), where \(\mu =x,~y,~z\),對應到線性代數就是一個一維的 array,向量也可視為一階張量。有兩個 indices 的量稱為二階張量,寫成 \(B_{\mu\nu}\), where \(\mu =x,~y,~z\), \(\nu =x,~y,~z\), 對應到矩陣。以 index 來表示的好處是,可以表達更高維的量,如 \(C_{ijk}\) (張量裡的下標都可以是 \(x\), \(y\), \(z\))。此外,\sphinxstylestrong{用 index 表示許多可能性,也可以簡化向量運算}。

\sphinxAtStartPar
常用來當張量 index 的字母有 \(\mu\), \(\nu\), \(i\), \(j\), \(k\), \(l\), \(m\), \(n\), \(\alpha\), \(\beta\)。 當然也可以用其他字母。


\begin{savenotes}\sphinxattablestart
\centering
\sphinxcapstartof{table}
\sphinxthecaptionisattop
\sphinxcaption{Tensor}\label{\detokenize{va/vec_alge:ten}}
\sphinxaftertopcaption
\begin{tabulary}{\linewidth}[t]{|T|T|T|T|}
\hline

\sphinxAtStartPar

&\sphinxstyletheadfamily 
\sphinxAtStartPar
number of indice
&\sphinxstyletheadfamily 
\sphinxAtStartPar
example
&\sphinxstyletheadfamily 
\sphinxAtStartPar
name
\\
\hline
\sphinxAtStartPar
零階張量
&
\sphinxAtStartPar
0
&
\sphinxAtStartPar
\(a\)
&
\sphinxAtStartPar
scalar 純量
\\
\hline
\sphinxAtStartPar
一階張量
&
\sphinxAtStartPar
1
&
\sphinxAtStartPar
\(A_\mu\)
&
\sphinxAtStartPar
vector 向量
\\
\hline
\sphinxAtStartPar
二階張量
&
\sphinxAtStartPar
2
&
\sphinxAtStartPar
\(B_{\mu\nu}\)
&
\sphinxAtStartPar
張量
\\
\hline
\sphinxAtStartPar
三階張量
&
\sphinxAtStartPar
3
&
\sphinxAtStartPar
\(C_{ijk}\)
&
\sphinxAtStartPar
張量
\\
\hline
\end{tabulary}
\par
\sphinxattableend\end{savenotes}
\begin{quote}

\sphinxAtStartPar
所以,當我們寫 \(A_\mu\)時,這個數就代表一個向量,因為\(\mu=x,~y,~z\) 有三種可能性
\end{quote}


\subsubsection{Einstein conventions}
\label{\detokenize{va/vec_alge:einstein-conventions}}
\sphinxAtStartPar
以張量符號處理線性代數的操作時,例如矩陣相乘,常常會遇到很多求和的情況,需要處理很多summations
\begin{equation*}
\begin{split}\sum_{j}A_{ij}B_{jk} \end{split}
\end{equation*}\begin{equation*}
\begin{split}\sum_{j}\sum_{k}A_{ij}B_{jk}C_{km}\end{split}
\end{equation*}
\sphinxAtStartPar
就需要寫很多次 \(\sum\)。為了簡化,我們就使用 Einstein convention
\label{va/vec_alge:definition-0}
\begin{sphinxadmonition}{note}{Definition 1.1 (Einstein convention)}



\sphinxAtStartPar
遇到張量相乘時,如果 index 重複出現,則自動視為有 summation,也就是
\begin{equation*}
\begin{split}A_{ij}B_{jk}=\sum_{j}A_{ij}B_{jk}\end{split}
\end{equation*}\begin{equation*}
\begin{split}A_{ij}B_{jk}C_{km}=\sum_{j}\sum_{k}A_{ij}B_{jk}C_{km}\end{split}
\end{equation*}\end{sphinxadmonition}
\phantomsection \label{exercise:exer_ten}

\begin{sphinxadmonition}{note}{Exercise 1.1}


\begin{enumerate}
\sphinxsetlistlabels{\arabic}{enumi}{enumii}{}{.}%
\item {} 
\sphinxAtStartPar
\(\mathbf{A} =(1,2,3)\)
\(\mathbf{B} =(6,2,5)\)
\begin{equation*}
\begin{split}A_\mu B_{\mu} =?\end{split}
\end{equation*}
\item {} \begin{equation*}
\begin{split} C_{ij} = \begin{pmatrix}
0&1&0\\
0&0&-1\\
1&0&0
\end{pmatrix}\end{split}
\end{equation*}
\end{enumerate}
\begin{equation*}
\begin{split}B_iC_{ij}A_j=?\end{split}
\end{equation*}\end{sphinxadmonition}
\phantomsection \label{va/vec_alge:solu_ten}

\begin{sphinxadmonition}{note}{Solution to Exercise 1.1}


\begin{enumerate}
\sphinxsetlistlabels{\arabic}{enumi}{enumii}{}{.}%
\item {} \begin{equation*}
\begin{split}A_\mu B_\mu =\sum_{\mu = x,y,z}A_\mu B_\mu = A_x B_x + A_y B_y+A_zB_z=25 \end{split}
\end{equation*}
\item {} \begin{equation*}
\begin{split}B_iC_{ij}A_j=\sum_{i}\sum_{j} B_iC_{ij}A_j = 11 \end{split}
\end{equation*}
\end{enumerate}
\end{sphinxadmonition}


\subsection{Free index, contraction}
\label{\detokenize{va/vec_alge:free-index-contraction}}
\sphinxAtStartPar
當張量有一個index時,我們可視它為一個向量,例如\(A_\mu\)是一個向量,因為這時候 \(\mu\) 是一個未指定的下標,
所以 \(A_\mu\) 可以有 \(A_x\), \(A_y\), \(A_z\) 三種可能性,因此我們稱 \(\mu\) 是 free index。
但是遇到\(A_\mu B_\mu\) 這種乘積時,我們必須使用 Einstein convention,因此必須把 \(\mu\)的所有可能列出以計算最後的和,因此 \sphinxstylestrong{\(A_\mu B_\mu\)} 這個數並沒有任何分量(可能性),是一個純量 scalar,此時,我們稱 \(\mu\) 為 dummy index。 這個概念與線性代數裡和的index \(\sum_{i=0,1,2,...}\) 一樣,此時 \(i\) 是dummy index,因為它已經使用在合的計算中。\sphinxstylestrong{對於計算而言,dummy index 的符號可以任意改變(例如 \(\sum_{i} = \sum_{j}\),可以隨意改變\(i\)成另外一個符號)},這也是它被稱為dummy的原因。


\chapter{Gallery of Electromagnetism}
\label{\detokenize{gallery:gallery-of-electromagnetism}}\label{\detokenize{gallery::doc}}
\begin{sphinxuseclass}{sd-tab-set}
\begin{sphinxuseclass}{sd-tab-item}\subsubsection*{v\sphinxhyphen{}dipole}

\begin{sphinxuseclass}{sd-tab-content}
\begin{figure}[htbp]
\centering
\capstart

\noindent\sphinxincludegraphics[height=450\sphinxpxdimen]{{dr}.png}
\caption{Vertical dipole radition.}\label{\detokenize{gallery:dr}}\end{figure}

\end{sphinxuseclass}
\end{sphinxuseclass}
\begin{sphinxuseclass}{sd-tab-item}\subsubsection*{v\sphinxhyphen{}dipole SPP}

\begin{sphinxuseclass}{sd-tab-content}
\begin{figure}[htbp]
\centering
\capstart

\noindent\sphinxincludegraphics[height=450\sphinxpxdimen]{{dr_plasmon}.png}
\caption{Vertical dipole radition excites SPPs.}\label{\detokenize{gallery:dr-plasmon}}\end{figure}

\end{sphinxuseclass}
\end{sphinxuseclass}
\begin{sphinxuseclass}{sd-tab-item}\subsubsection*{c\sphinxhyphen{}dipole}

\begin{sphinxuseclass}{sd-tab-content}
\begin{figure}[htbp]
\centering
\capstart

\noindent\sphinxincludegraphics[height=450\sphinxpxdimen]{{c_dr}.png}
\caption{Circular dipole radition.}\label{\detokenize{gallery:c-dr}}\end{figure}

\end{sphinxuseclass}
\end{sphinxuseclass}
\begin{sphinxuseclass}{sd-tab-item}\subsubsection*{c\sphinxhyphen{}dipole SPP}

\begin{sphinxuseclass}{sd-tab-content}
\begin{figure}[htbp]
\centering
\capstart

\noindent\sphinxincludegraphics[height=450\sphinxpxdimen]{{c_dr_plasmon}.png}
\caption{Circular dipole radition excites SPPs.}\label{\detokenize{gallery:c-dr-plasmon}}\end{figure}

\end{sphinxuseclass}
\end{sphinxuseclass}
\end{sphinxuseclass}





\renewcommand{\indexname}{Proof Index}
\begin{sphinxtheindex}
\let\bigletter\sphinxstyleindexlettergroup
\bigletter{definition\sphinxhyphen{}0}
\item\relax\sphinxstyleindexentry{definition\sphinxhyphen{}0}\sphinxstyleindexextra{va/vec\_alge}\sphinxstyleindexpageref{va/vec_alge:\detokenize{definition-0}}
\end{sphinxtheindex}

\renewcommand{\indexname}{Index}
\printindex
\end{document}